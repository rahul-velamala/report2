\chapter{Literature Review}
\label{chap:literature_review}
\section{Overview of Decentralized Multi-Agent Path Planning}
This chapter presents a comprehensive review of the methods and frameworks used in modern multi-agent path planning (MAPF) systems, especially those based on decentralized architectures.
It investigates various approaches to path planning in multi-robot systems, highlights the evolution towards decentralized and dynamically adaptive strategies, and discusses the increasing relevance of decentralized MAPF in complex and uncertain environments. Each section explores key concepts, technological advancements, and inherent challenges in developing and implementing decentralized MAPF solutions. The chapter aims to outline the state of the art, identify existing research gaps, and position the contributions of this work within the broader field.

\section{Multi-Agent Path Planning System Architectures}
A multi-agent path planning system can be broadly categorized based on its architecture, which dictates information flow and decision-making processes:
\begin{itemize}
    \item \textbf{Centralized Architectures:} In centralized systems, a single planning unit possesses complete knowledge of all agents, their goals, and the environment. This unit computes paths for all agents, ensuring coordination and often aiming for global optimality. Algorithms such as M* and CBS are key examples of this approach.
    \item \textbf{Decentralized Architectures:} Decentralized systems distribute planning and decision-making among individual agents. Each agent plans its path based on local information and limited communication with neighbors. Methods like VO and ORCA are representative of decentralized strategies.
    \item \textbf{Coupled Approaches:} Coupled approaches, often found in centralized systems, consider the joint state space of all agents simultaneously. This allows for optimal solutions but suffers from scalability issues. CBS and M* are examples of coupled approaches.
    \item \textbf{Decoupled Approaches:} Decoupled approaches simplify planning by considering each agent's path individually, addressing conflicts reactively. VO and ORCA are examples of decoupled methods that prioritize computational efficiency.
    \item \textbf{Networked Topologies:} Multi-agent systems often operate within networked topologies, where communication and coordination are constrained by the network structure. Graph-based representations, like those used in GNNs and algorithms like PRIMAL, are well-suited to model and exploit these topologies for decentralized MAPF.
\end{itemize}

\section{Centralized Multi-Agent Path Planning Techniques}
Centralized MAPF approaches, while offering optimality guarantees under certain conditions, face scalability limitations and single points of failure. This section reviews key centralized techniques and their characteristics.

\subsection{Optimal Centralized Algorithms: M* and ICTS}
Optimal centralized algorithms like M* \cite{Standley2011Complete}
and ICTS (Increasing Cost Tree Search) aim to find solutions with the minimum total cost (e.g., sum of path lengths) for all agents. M* guarantees optimality and completeness by systematically exploring the joint state space of all agents. ICTS, while also striving for optimality, improves efficiency over basic search algorithms by focusing on conflict resolution within a tree-based search framework. However, the computational complexity of these algorithms grows exponentially with the number of agents and environment size, limiting their applicability to large-scale problems.

\subsection{Conflict-Based Search (CBS)}
Conflict-Based Search (CBS) \cite{Sharon2015CBS}
is a prominent centralized MAPF algorithm that efficiently finds optimal or near-optimal solutions by iteratively resolving conflicts between agents paths. CBS operates on a two-level search structure: a high-level search that identifies and resolves conflicts by adding constraints, and a low-level search (typically A*) that replans paths for individual agents under these constraints. CBS significantly improves scalability compared to brute-force search methods but still faces computational challenges with increasing agent density and complex environments. Despite its efficiency improvements over purely exhaustive searches, CBS remains computationally expensive for very large teams or highly complex environments.

\section{Decentralized Multi-Agent Path Planning Techniques}
Decentralized MAPF methods prioritize scalability and robustness by distributing planning and decision-making among individual agents. This section reviews key decentralized approaches, including velocity obstacle based methods and learning-based techniques.

\subsection{Velocity Obstacle (VO) and Optimal Reciprocal Collision Avoidance (ORCA)}
Velocity Obstacle (VO) \cite{VanDenBerg2008ORCA}
and its refinement, Optimal Reciprocal Collision Avoidance (ORCA), are classic decentralized approaches for collision avoidance. VO defines a velocity obstacle for each agent based on the relative velocities and positions of neighboring agents. ORCA improves upon VO by ensuring reciprocal collision avoidance and smoother trajectories. These methods are computationally efficient and reactive, enabling real-time collision avoidance in dynamic environments. However, VO and ORCA are primarily reactive and may not guarantee goal achievement or optimality in complex scenarios, and can sometimes lead to local minima or deadlocks in tightly constrained spaces.

\subsection{Decentralized Planning with Graph Neural Networks (GNNs) and Adaptive Communication}
Recent advancements explore learning-based decentralized MAPF, particularly utilizing Graph Neural Networks (GNNs) and Reinforcement Learning (RL). GNNs, as demonstrated by Li et al. \cite{Li2021GNNCoordination},
are well-suited for decentralized MAPF due to their ability to model agent interactions within a graph structure, facilitating distributed information aggregation and decision-making. PRIMAL (Pathfinding via Reinforcement and Imitation Multi-Agent Learning) \cite{Sartoretti2019Primal} combines imitation learning from expert demonstrations with reinforcement learning to train decentralized policies, often without explicit communication modeling.

While GNNs show promise, standard approaches often rely on aggregating information from a \textbf{fixed K-hop neighborhood}. This means each agent communicates and integrates information only from peers within a predefined number of communication steps (hops). This fixed structure presents a limitation: the optimal communication range might vary depending on the environment density, the number of robots, or the specific tactical situation, and a single fixed K may not be universally optimal. Selecting the best K often requires manual tuning or heuristics, potentially limiting the adaptability and peak performance of the learned policy.

To address the limitations of fixed communication ranges, adaptive graph methods like Adaptive Diffusion Convolution (ADC) \cite{Zhao2021ADC} have been proposed in the broader GNN literature. ADC replaces discrete K-hop aggregation with a learnable diffusion process, characterized by a continuous radius parameter $t$. This parameter $t$, representing the effective extent of information propagation, is automatically learned during training, allowing the network itself to determine the most appropriate neighborhood size for feature aggregation based on the task objective. Integrating such adaptive diffusion mechanisms into GNNs for MAPF holds the potential to improve coordination policies by dynamically adjusting the communication neighborhood size, enhancing adaptability compared to fixed-K approaches.

\section{Relevance of Decentralized Multi-Agent Path Planning Systems}
Decentralized MAPF systems are increasingly relevant due to their inherent advantages in various robotic applications:
\begin{itemize}
    \item \textbf{Scalability:} Decentralized approaches scale more effectively to large teams of robots, as computational load is distributed across agents rather than concentrated in a central unit.
    \item \textbf{Robustness and Fault Tolerance:} Decentralized systems are more robust to failures, as the system can continue operating even if individual agents or communication links fail.
    \item \textbf{Adaptability to Dynamic Environments:} Decentralized agents can react more quickly to changes in the environment, as they rely on local sensing and communication rather than waiting for global updates.
    \item \textbf{Real-time Performance:} Decentralized algorithms are often more computationally efficient, enabling real-time path planning and execution, crucial for dynamic applications.
    \item \textbf{Versatility and Flexibility:} Decentralized systems can be deployed in diverse environments, including those with limited communication infrastructure or unknown topologies.
    \item \textbf{Cost-Effectiveness:} Decentralized systems can reduce reliance on expensive centralized computing and communication infrastructure, potentially lowering overall system costs.
    \item \textbf{Applications in Complex Domains:} Decentralized MAPF is essential for applications in domains like warehouse automation, autonomous driving, search and rescue, and space exploration, where large teams of robots must operate autonomously in dynamic and uncertain conditions.
\end{itemize}

\section{Challenges and Research Gaps in Decentralized MAPF}
\label{sec:challenges_gaps}
Despite their advantages, Decentralized MAPF systems face several significant challenges that represent active areas of research:
\begin{itemize}
    \item \textbf{Sub-optimality:} Decentralized solutions may not guarantee global optimality, as agents make decisions based on limited local information.
    \item \textbf{Deadlocks and Local Minima:} Insufficient coordination in decentralized systems can lead to deadlocks or agents becoming trapped in local minima.
    \item \textbf{Partial Observability:} Agents typically have only partial views of the environment, making it difficult to plan globally optimal paths and anticipate the actions of distant agents.
    \item \textbf{Communication Constraints and Strategy:}
        \begin{itemize}
            \item Limited communication range, bandwidth, or reliability can hinder effective coordination.
            \item \textbf{Gap:} Determining the optimal range and content of information to share remains a challenge. Specifically, most learning-based approaches, particularly GNNs, rely on \textit{fixed, predefined communication ranges (e.g., K-hops)}, which may not be suitable for diverse or dynamic scenarios. There is a need for mechanisms that allow the communication range to be learned or adapted.
        \end{itemize}
    \item \textbf{Dynamic and Uncertain Environments:} Adapting to dynamic obstacles, changing goals, and noisy sensor data in real-time remains a significant challenge.
    \item \textbf{Coordination Complexity:} Designing effective coordination mechanisms, especially for complex tasks and dense agent populations, is non-trivial.
    \item \textbf{Verification and Validation:} Verifying the correctness and safety of decentralized MAPF algorithms, particularly in safety-critical applications, can be challenging.
    \item \textbf{Adaptability and Learning in Decentralized Settings:}
        \begin{itemize}
            \item Enabling decentralized agents to learn robust policies that generalize well to unseen environments and tasks is an ongoing research area.
            \item \textbf{Gap:} While GNNs provide a framework for learning, their adaptability can be hampered by rigid structural assumptions, such as the aforementioned fixed communication ranges. Enhancing this adaptability is crucial.
        \end{itemize}
    \item \textbf{Balancing Reactivity and Proactiveness:} Decentralized agents need to be reactive to local changes while also exhibiting proactive behavior to achieve long-term goals and avoid potential future conflicts.
    \item \textbf{Ensuring Fairness and Efficiency:} In multi-agent systems, ensuring fairness in resource allocation and efficient overall system performance requires careful design of decentralized algorithms.
\end{itemize}

\subsection{Research Gaps Addressed by This Work}
\begin{enumerate}
    \item \textbf{Fixed Communication Range in GNN-based MAPF:} The primary gap tackled is the reliance of existing GNN-based decentralized MAPF frameworks (e.g., \cite{Li2021GNNCoordination}) on a fixed K-hop neighborhood for information aggregation. This work introduces an \textit{adaptive communication mechanism} by integrating Adaptive Diffusion Convolution (ADC) \cite{Zhao2021ADC}. This allows the effective communication range (diffusion radius $t$) to be learned automatically during training, rather than being manually predefined.
    \item \textbf{Limited Adaptability of Learned Policies:} By enabling the learning of an adaptive communication range, this work seeks to improve the adaptability of the learned coordination policies. The hypothesis is that allowing the network to determine the optimal extent of information propagation will lead to policies that perform more robustly across diverse environmental conditions (e.g., varying obstacle densities) and generalize better to unseen scenarios compared to models with fixed communication ranges.
\end{enumerate}
While there are still many challenges like finding the best possible solutions or dealing with fast-changing environments, this thesis mainly focuses on improving how robots communicate and adapt in learning-based decentralized MRPP.

\section{Chapter Summary}
This chapter has provided a literature review focusing on Decentralized Multi-Agent Path Planning. We explored the spectrum of MAPF approaches, from centralized optimal algorithms like M* and CBS to decentralized reactive methods like VO and ORCA, as well as emerging learning-based techniques utilizing GNNs, such as those by Li et al. \cite{Li2021GNNCoordination}, and RL-based methods like PRIMAL \cite{Sartoretti2019Primal}. A key challenge and research gap identified in current GNN-based learning methods is their reliance on fixed communication ranges, which can limit performance and adaptability across varied scenarios.

The chapter highlighted the relevance and advantages of decentralized MAPF in complex robotic applications, while also outlining the significant general challenges that remain. This thesis investigates the research gap concerning fixed communication ranges in GNN-based MAPF by exploring the use of Adaptive Diffusion Convolution (ADC) \cite{Zhao2021ADC} as a means to enable learning of effective and adaptive communication patterns. In the following chapters, we study how such adaptive communication mechanisms can influence the performance, robustness, and generalization capabilities of decentralized MAPF systems.